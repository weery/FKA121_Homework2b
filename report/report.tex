\input{template_files/packages}

\input{settings}

\title{H2b: Variational Monte Carlo}
\author{Victor Nilsson and Simon Nilsson}
\date{\today}

\begin{document}

\input{template_files/titlepage}

\section*{Introduction}



\section*{Problem 1}

\begin{figure}[H]
    \centering
    \captionsetup[subfigure]{justification=centering}
    \begin{subfigure}[b]{0.40\textwidth}
        \centering
        \includegraphics[width=\textwidth]{graphics/task1/radius.png}
    \end{subfigure}
    \begin{subfigure}[b]{0.40\textwidth}
        \centering
        \includegraphics[width=\textwidth]{graphics/task1/angle_diff_dist.png}
    \end{subfigure}
    \caption{Simulated and calculated radial distribution function for the two electrons in the helium atom. The distribution of the relative $\theta$ angle for the two electrons.}
    \label{fig:RDF}
\end{figure}

As we can see in (Fig.~\ref{fig:RDF}) we can see that out radial distribution looks like the variationally optimized distribution.

\section*{Problem 2}

\begin{figure}[H]
	\centering
	\captionsetup[subfigure]{justification=centering}
	\begin{subfigure}[b]{0.4\textwidth}
		\centering
		\includegraphics[width=\textwidth]{graphics/task2/block_error.png}
	\end{subfigure}
	\caption{Estimation of the statistical inefficiency based on different block lengths}
	\label{fig:block_error}
\end{figure}



\section*{Problem 3}


\begin{figure}[H]
	\centering
	\captionsetup[subfigure]{justification=centering}
	\begin{subfigure}[b]{0.4\textwidth}
		\centering
		\includegraphics[width=\textwidth]{graphics/task3/lowest_energy.png}
	\end{subfigure}
	\caption{Estimation of the statistical inefficiency based on different block lengths}
	\label{fig:optimize_alpha}
\end{figure}


\section*{Problem 4}

\begin{figure}[H]
    \centering
    \captionsetup[subfigure]{justification=centering}
    \begin{subfigure}[b]{0.40\textwidth}
        \centering
        \includegraphics[width=\textwidth]{graphics/task4/alpha_estimate.png}
    \end{subfigure}
    \begin{subfigure}[b]{0.40\textwidth}
        \centering
        \includegraphics[width=\textwidth]{graphics/task4/lowest_energy.png}
	\end{subfigure}
    \caption{dummy text}
    \label{fig:alpha_estimation}
\end{figure}

\begin{equation}
	\psi_t(r_1,r_2) = e^{-2r_1}e^{-2r_2}e^{\frac{r_{12}}{2(1+\alpha r_{12})}}
\end{equation}

\begin{equation}
	\nabla \alpha \ln{\psi_t(r_1,r_2)} = -\frac{r_{12}^2}{2(1+\alpha r_{12})^2}
\end{equation}

$\beta = 0.75$
$\alpha_{min}=0.142553129000000$
$E_{min}=-2.891044220000000$

\section*{Problem 5}

$E_min=-2.878146$


\appendix
\section{Source code}

\subsection{\texttt{Task1/main.c}}
%\lstinputlisting[language=c, numbers=left]{../code/Task1/main.c}

\end{document}
