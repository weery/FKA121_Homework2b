\input{template_files/packages}

\input{settings}

\title{H2b: Variational Monte Carlo}
\author{Victor Nilsson and Simon Nilsson}
\date{\today}

\begin{document}

\input{template_files/titlepage}

\section*{Introduction}



\section*{Problem 1}

\begin{figure}[H]
    \centering
    \captionsetup[subfigure]{justification=centering}
    \begin{subfigure}[b]{0.40\textwidth}
        \centering
        \includegraphics[width=\textwidth]{graphics/task1/radius.png}
    \end{subfigure}
    \begin{subfigure}[b]{0.40\textwidth}
        \centering
        \includegraphics[width=\textwidth]{graphics/task1/angle_diff_dist.png}
    \end{subfigure}
    \caption{\textit{Left:} Simulated and calculated radial distribution function for the two electrons in the helium atom. \textit{Right:} The distribution of cosine of the relative $\theta$ angle for the two electrons.}
    \label{fig:RDF}
\end{figure}

In this problem we studied a He atom using the variational Monte Carlo method with the Metropolis algorithm. In the Metropolis algorithm a new trial state is obtained by randomly displacing both electrons in all three dimensions simultaneously. The displacement is bounded by a cube with side $d$, where $d$ is a simulation parameter which was chosen as 1 \AA. Other equations used can be found in \cite{probdesc}.

From the obtained sample points, a radial probability distribution function could be approxiated, which can be seen in Fig.~\ref{fig:RDF} (left). Comparing the result to the variationally optimized distribution and the central field approximation result we conclude that that the results are the same as the variationally optimized distribution and that the method is working properly.

Transforming these sample points to spherical coordinates we can study the correlation of the $\theta$ coordinates for the two electron. As stated in \cite{probdesc}, if they are uncorrelated, i.e. equally distributed over the unit sphere, then the probability distribution would be $P(\Delta \theta) = \sin (\Delta \theta) / 2$ for $0 < \Delta\theta < \pi$, or
\begin{equation}
	x = \cos \left|\theta_1 - \theta_2\right| \Rightarrow P(x) = \frac{1}{2},\quad -1 < x < 1.
\end{equation}

From the variational Monte Carlo we obtain the result in Fig.~\ref{fig:RDF}. This indicates a clear correlation where the $\theta$ coordinate of the electrons tends to be approximately the same. At least a slight correlation is expected, however whether it should be this strongly correlated or not, we do not know.



\section*{Problem 2}

\begin{figure}[H]
	\centering
	\captionsetup[subfigure]{justification=centering}
	\begin{subfigure}[b]{0.4\textwidth}
		\centering
		\includegraphics[width=\textwidth]{graphics/task2/block_error.png}
	\end{subfigure}
	\caption{Estimation of the statistical inefficiency based on different block lengths}
	\label{fig:block_error}
\end{figure}


\section*{Problem 3}


\begin{figure}[H]
	\centering
	\captionsetup[subfigure]{justification=centering}
	\begin{subfigure}[b]{0.4\textwidth}
		\centering
		\includegraphics[width=\textwidth]{graphics/task3/lowest_energy.png}
	\end{subfigure}
	\caption{Estimation of the statistical inefficiency based on different block lengths}
	\label{fig:optimize_alpha}
\end{figure}



\section*{Problem 4}

\begin{equation}
	\psi_t(r_1,r_2) = e^{-2r_1}e^{-2r_2}e^{\frac{r_{12}}{2(1+\alpha r_{12})}}
\end{equation}

\begin{equation}
	\nabla \alpha \ln{\psi_t(r_1,r_2)} = -\frac{r_{12}^2}{2(1+\alpha r_{12})^2}
\end{equation}

$\beta = 0.75$
$\alpha_{min}=0.142553129000000$
$E_{min}=-2.891044220000000$

\section*{Problem 5}

$E_min=-2.878146$


\bibliography{references}
\bibliographystyle{plain}

\appendix
\section{Source code}

\subsection{\texttt{Task1/main.c}}
%\lstinputlisting[language=c, numbers=left]{../code/Task1/main.c}

\end{document}
